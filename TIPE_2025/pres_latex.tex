\documentclass{beamer}
\usepackage{bussproofs}
\usepackage{graphicx}
\usepackage{hyperref}
\usepackage{appendixnumberbeamer}
\usepackage{tikz}
\usetikzlibrary{fit}
\usepackage{forest}
\usetikzlibrary{shapes.rounded, arrows.meta, positioning}
\date{}


% Styles des bulles et flèches
\tikzstyle{bulle} = [
  rounded rectangle,
  draw=blue!70,
  fill=blue!10,
  thick,
  minimum height=1.9cm,
  text centered,
  align=center,          % ← permet les sauts de ligne
  font=\normalize,
  text width=2.5cm         % ← limite la largeur (important si tu veux forcer des sauts de ligne naturels)
]
\tikzstyle{fleche} = [-{Latex[length=3mm]}, thick]

\tikzstyle{bulle2} = [
  rounded rectangle,
  draw=blue!70,
  fill=blue!10,
  thick,
  minimum height=1.9cm,
  text centered,
  align=center,          % ← permet les sauts de ligne
  font=\normalize,
  text width=5cm         % ← limite la largeur (important si tu veux forcer des sauts de ligne naturels)
]




\setbeamertemplate{footline}[frame number]

\title{Formalisation des démonstrations pour l'étude des possibilités informatiques d'assistant de preuves.}
\author{Florian DEVARENNE \texorpdfstring{\\}{, }
       n°45154
       }     

\begin{document}

\begin{frame}
    \titlepage
\end{frame}

\section{Introduction}

\begin{frame}{Introduction}
    \begin{minipage}{0.48\textwidth}
        \centering
        \includegraphics[width=\textwidth]{Kurt-Gödel}
        \caption{Kurt Gödel}
    \end{minipage}
    \hfill
    \begin{minipage}{0.44\textwidth}
        \centering
        \includegraphics[width=\textwidth]{Gerhard_Gentzen}
        \caption{Gerhard Gentzen}
    \end{minipage} 
\end{frame}



\begin{frame}{Origine de ma réflexion}

\begin{center}
\resizebox{1.08\textwidth}{!}{ % <-- redimensionne tout le schéma à 95% de la largeur
\begin{tikzpicture}[node distance=1.8cm and 1.5cm]

% Ligne principale horizontale
\node[bulle] (godel) {Théorèmes d'incomplétudes\\ de Gödel};
\node[bulle, right=of godel] (limites) {Limites des assistants};
\node[bulle, right=of limites] (creation) {Créer un assistant ?};

% Bulles en dessous
\node[bulle, below left=of creation] (formalisation) {Formalisation des \\ démonstrations};
\node[bulle, below=of creation] (verificateur) {Vérificateur de \\ preuves};
\node[bulle, below right=of creation] (chercheur) {Chercheur de \\ preuves};

% Flèches
\draw[fleche] (godel) -- (limites);
\draw[fleche] (limites) -- (creation);
\draw[fleche] (creation) -- (formalisation);
\draw[fleche] (creation) -- (verificateur);
\draw[fleche] (creation) -- (chercheur);

\end{tikzpicture}
}
\end{center}

\end{frame}

\begin{frame}{Problématique et Ancrage au thème}
  \begin{block}{Problématique}
    Comment formaliser et implémenter un moteur de vérification de preuve en \textbf{déduction naturelle} en \texttt{OCaml}, tout en facilitant la compréhension des preuves grâce à une \textbf{traduction en langage naturel} ?
  \end{block}

    \vspace{0.8cm}

  \begin{block}{Lien avec le thème}
    \textbf{Transition}, \textbf{Transformation}, \textbf{Conversion} :
    \begin{itemize}
      \item Formalisation d'une preuve 
      \item Transformation d'un arbre de preuve en preuve en français (traducteur)
    \end{itemize}
  \end{block}
\end{frame}



\begin{frame}{Sommaire}
    \tableofcontents
\end{frame}


\begin{frame}
  \centering
  \vfill
  {\LARGE \textbf{ Préliminaires : Déduction naturelle et Choix d'implémentation}}
  \vfill
\end{frame}



\section{P - Déduction Naturelle}

\subsection{Présentation du système et vocabulaire}


\begin{frame}{La déduction naturelle}
    \begin{block}{Déduction naturelle}
    \begin{itemize}
        \item système formalisant les démonstrations
        \item reposant sur des règles de déduction
    \end{itemize}
    
    \end{block}

    \begin{block}{Pour mon projet en particulier}
    \begin{itemize}
        \item Calcul propositionnel (pas de {& \forall&} ,  {& \exists&})
    \end{itemize}
    
    \end{block}

    


    \vspace{2cm}

    \begin{minipage}{0.7\textwidth}
        \centering
        \includegraphics[width=\textwidth]{illu}
    \end{minipage}

\end{frame}


\begin{frame}{Une règle et un séquent en détail}

    \begin{itemize}
        \item Une règle d'élimination et d'introduction pour {$\land , \lor ,\neg , \to$}
        \item Une règle pour l'absurdité et l'affaiblissement
    \end{itemize}
  

    \begin{prooftree}
        \RightLabel{($\to$ i)}
        \AxiomC{$\Gamma , A \vdash B$}
        \UnaryInfC{$\Gamma \vdash A \to B$}
    \end{prooftree}

    \vspace{0.5cm}

    \begin{prooftree}
        \RightLabel{($\to$ e)}
        \AxiomC{$\Gamma \vdash A$}
        \AxiomC{$\Gamma \vdash A \to B$}
        \BinaryInfC{$\Gamma \vdash B$}
    \end{prooftree}

    \vspace{1cm}

    Un séquent : 
    \begin{prooftree}
        
    
    {$\Gamma \vdash A \to B$}

    \end{prooftree}

    
\end{frame}




\begin{frame}{un exemple de preuve courte}

    \begin{prooftree}       
        \RightLabel{(Ax)}
        \AxiomC{}
        \UnaryInfC{$\neg A,A \vdash A$}
        \RightLabel{(Ax)}
        \AxiomC{}
        \UnaryInfC{$\neg A , A \vdash \neg A$}
        \RightLabel{($\neg$ e)}
        \BinaryInfC{$\neg A , A \vdash \bot$}
        \RightLabel{($\to$ i)}
        \UnaryInfC{$\neg A \vdash (A \to \bot)$}
        \RightLabel{($\to$ i)}
        \UnaryInfC{$\vdash\ \neg A \to (A \to \bot)$}
        
    \end{prooftree}

\end{frame}


\subsection{Implémentation du système en OCaml}

\begin{frame}{Choix d'implémentation}

    \begin{minipage}{0.48\textwidth}
        \begin{itemize}
            \item Type enregistrement/structuré pour sequent et regle
            \vspace{0.4cm}
            \item Type somme pour Formule
            \vspace{0.4cm}
            \item Structure d'arbre pour demonstration
        \end{itemize}
    \end{minipage}
    \hfill
    \begin{minipage}{0.44\textwidth}
        \centering
        \includegraphics[width=\textwidth]{deftype}
    \end{minipage} 
    
    
\end{frame}



\section{I - Vérificateur de preuve}


\begin{frame}
  \centering
  \vfill
  {\LARGE \textbf{Objectif 1: Vérificateur de preuve}}
  \vfill
\end{frame}


\begin{frame}{Première objectif : vérificateur de preuve}

    On veut construire une fonction OCaml de signature demonstration {& \to &} bool


    \begin{block}{Pour y arriver : }
    \begin{itemize}
        \item Quelle règle doit respecter une preuve ?
        \item Vérifier ces points facilement en fonction de nos choix d'implémentation
    \end{itemize}
    
    \end{block}

    



\end{frame}

\subsection{Principe de fonctionnement}








\begin{frame}{Principe de fonctionnement}

    \begin{block}{Vérification de validité}
        \begin{itemize}
            \item Pour chaque lien entre les règles : conclusion de la règle du dessus correspond à l'une des prémisses de la règle d'en dessous
            \item Les feuilles de l'arbre sont des "Axiomes"
            \item Précondition : utilisation des règles implémentées uniquement
        \end{itemize}
    \end{block}

    \vspace{0.8cm}

    \begin{minipage}{1.05\textwidth}
        \centering
        \includegraphics[width=\textwidth]{verificateur}
    \end{minipage}
    
\end{frame}







\begin{frame}{Visualisation de l'arbre de preuve}

\begin{center}
\resizebox{1.05\textwidth}{!}{ % Ajuste la taille du schéma
\begin{tikzpicture}[node distance=1.8cm and 2cm]


% Noeuds
\node[bulle2] (regle1) {
    \begin{prooftree}
        \RightLabel{(intro et)}
        \AxiomC{$A,B \vdash B$}
        \AxiomC{$A,B \vdash A$}
        \BinaryInfC{$A,B \vdash A \land B$}
    \end{prooftree}
};



\node[bulle, above left=of regle1] (regle2) {
    \begin{prooftree}
        \RightLabel{(Ax)}
        \AxiomC{}
        \UnaryInfC{$\ A,B \vdash A$}
    \end{prooftree}
};
\node[bulle, above right=of regle1] (regle3) {
    \begin{prooftree}
        \RightLabel{(Ax)}
        \AxiomC{}
        \UnaryInfC{$\ A,B \vdash B$}
    \end{prooftree}
};

% Flèches vers le haut
\draw[fleche] (regle1) -- (regle2);
\draw[fleche] (regle1) -- (regle3);

\end{tikzpicture}
}
\end{center}

\end{frame}


\subsection{Exemple d'utilisation}

\begin{frame}{Test de notre verificateur sur nos exemples}

    \begin{block}{Vérification de la preuve des séquents : }
    \begin{itemize}
        \item {&A, B  \vdash  A \land B   &}
        
        \item {$\vdash\ \neg A \to (A \to \bot)$}  
    \end{itemize}
        
    \end{block}

    \vspace{1cm}

    \begin{block}{Résultat des tests}


    \begin{minipage}{1.05\textwidth}
        \centering
        \includegraphics[width=\textwidth]{resultat}
    \end{minipage}      
    \end{block}


    
\end{frame}


\section{II - Traducteur/Assistant de preuve}


\begin{frame}
  \centering
  \vfill
  {\LARGE \textbf{Objectif 2: Traducteur/Assistant de preuve}}
  \vfill
\end{frame}



\begin{frame}{Deuxième objectif : traducteur/assistant de preuve}



    On veut construire une fonction OCaml de signature \\sequent {$ \to $} bool

    \vspace{1cm}

    \begin{block}{Pour y arriver : }
    \begin{itemize}
        \item aide l'utilisateur à créer une preuve étape par étape
        \item traduction en français des regles 
        \item implémenter afin de facilité la vérification par notre programme précédent
        \end{itemize}
    
    \end{block}

    


    
\end{frame}

\subsection{Principe de fonctionnement}

\begin{frame}{Principe de fonctionnement}

    \begin{block}{Découpage en étape}
    \begin{itemize}
        \item Fonction construction\_demo : sequent {$\to$} demonstration
        \item Fonction assistant ci-dessous
    \end{itemize}
        
    \end{block}


    \vspace{1.5cm}

    \begin{minipage}{1.05\textwidth}
        \centering
        \includegraphics[width=\textwidth]{assis2}
    \end{minipage} 

         

\end{frame}


\begin{frame}{Fonctionnement pratique}

    \begin{minipage}{1.05\textwidth}
        \centering
        \includegraphics[width=\textwidth]{forme_assis}
    \end{minipage} 
    
    
\end{frame}

\subsection{Exemple d'utilisation}


\begin{frame}{Test sur exemple}
    \begin{block}{On montre le séquent {$A , B \vdash A \land B$}}
    \begin{itemize}
        \item On commence par choisir la règle {$\land i$}
        \item Le programme nous demande donc de prouver {$A , B \vdash A$} puis {$A, B \vdash B$}
        \item On prouve ces deux séquents avec axiome 
        \item Fin, on obtient le résultat : 
    \end{itemize}
  
    \end{block}

    \begin{minipage}{1.05\textwidth}
        \centering
        \includegraphics[width=\textwidth]{result_assis}
    \end{minipage} 
\end{frame}

\begin{frame}{Test sur exemple}
    \begin{block}{La preuve en français donné par notre programme}
        On veut montrer {$A \land B$} en supposant B , A or B , A permet individuellement de montrer A et aussi B donc on a {$A \land B$} \\
        On veut montrer A en supposant B ,  A 
        puis ce qu'on le suppose, on a bien A \\
        On veut montrer B en supposant B ,  A 
        puis ce qu'on le suppose, on a bien B 


        
    \end{block}
    
\end{frame}


\section{Conclusion}


\begin{frame}
  \centering
  \vfill
  {\LARGE \textbf{Conclusion et Ouverture}}
  \vfill
\end{frame}



\begin{frame}{Les assistants de nos jours}

    
    \begin{itemize}
        \item Coq, Isabelle , Lean
        \item Méthode complexe : Tableaux, Unifications
        \item IA ?
        \item Utilisation 
    \end{itemize}
    
\end{frame}


\begin{frame}{Retours et Critiques}

    \begin{block}{Points positifs}
        \begin{itemize}
            \item Résultats satisfaisants
            \item Fonctionne pour tout séquent de calcul propositionnel
        \end{itemize}        
    \end{block}
    \vspace{1cm}
    \begin{block}{Axes d'amélioration}
        \begin{itemize}
            \item Meilleure interface utilisateur
            \item Aide pour la preuve autre que traduction
            \item Implémentation moins lourde
        \end{itemize}
        
    \end{block}
    
    
    
\end{frame}









\begin{frame}{Conclusion}
    \begin{flushright}
        \textit{« Pour rêver à l'infini des nombres, il faudra toujours des mathématiciennes et des mathématiciens. »}\\
         \vspace{1cm}
        \small{- Arte, Voyage au pays des maths , L' Entscheidungsproblem ou la fin des mathématiques ?}
    \end{flushright}
\end{frame}


\section{Annexe}

\appendix



\begin{frame}
  \centering
  \vfill
  {\LARGE \textbf{Annexe : code et tests}}
  \vfill
\end{frame}


\begin{frame}{Règle p1}
    \begin{minipage}{0.95\textwidth}
        \centering
        \includegraphics[width=\textwidth]{regle1}
    \end{minipage} 
    
\end{frame}

\begin{frame}{Règle p2}
    \begin{minipage}{0.95\textwidth}
        \centering
        \includegraphics[width=\textwidth]{regle2}
    \end{minipage} 
    
\end{frame}


\begin{frame}{Test de l'assistant p1}
    \begin{minipage}{0.95\textwidth}
        \centering
        \includegraphics[width=\textwidth]{test1}
    \end{minipage} 
    
\end{frame}


\begin{frame}{Test de l'assistant p2}
    \begin{minipage}{0.95\textwidth}
        \centering
        \includegraphics[width=\textwidth]{test2}
    \end{minipage} 
    
\end{frame}


\begin{frame}{Test de l'assistant p3}
    \begin{minipage}{0.95\textwidth}
        \centering
        \includegraphics[width=\textwidth]{test3}
    \end{minipage} 
    
\end{frame}

\begin{frame}{Test de l'assistant p4}
    \begin{minipage}{0.95\textwidth}
        \centering
        \includegraphics[width=\textwidth]{test4}
    \end{minipage} 
    
\end{frame}

\begin{frame}{Test de l'assistant p5}
    \begin{minipage}{0.95\textwidth}
        \centering
        \includegraphics[width=\textwidth]{test5}
    \end{minipage} 
    
\end{frame}

\begin{frame}{Test de l'assistant p6}
    \begin{minipage}{0.95\textwidth}
        \centering
        \includegraphics[width=\textwidth]{test6}
    \end{minipage} 
    
\end{frame}


\begin{frame}{Code p1}
    \begin{minipage}{0.95\textwidth}
        \centering
        \includegraphics[width=\textwidth]{code1}
    \end{minipage} 
    
\end{frame}

\begin{frame}{Code p2}
    \begin{minipage}{0.95\textwidth}
        \centering
        \includegraphics[width=\textwidth]{code2}
    \end{minipage} 
    
\end{frame}

\begin{frame}{Code p3}
    \begin{minipage}{0.95\textwidth}
        \centering
        \includegraphics[width=\textwidth]{code3}
    \end{minipage} 
    
\end{frame}

\begin{frame}{Code p4}
    \begin{minipage}{0.95\textwidth}
        \centering
        \includegraphics[width=\textwidth]{code4}
    \end{minipage} 
    
\end{frame}

\begin{frame}{Code p5}
    \begin{minipage}{0.95\textwidth}
        \centering
        \includegraphics[width=\textwidth]{code5}
    \end{minipage} 
    
\end{frame}

\begin{frame}{Code p6}
    \begin{minipage}{0.95\textwidth}
        \centering
        \includegraphics[width=\textwidth]{code6}
    \end{minipage} 
    
\end{frame}

\begin{frame}{Code p7}
    \begin{minipage}{0.95\textwidth}
        \centering
        \includegraphics[width=\textwidth]{code7}
    \end{minipage} 
    
\end{frame}

\begin{frame}{Code p8}
    \begin{minipage}{1.2\textwidth}
        \centering
        \includegraphics[width=\textwidth]{code8}
    \end{minipage} 
    
\end{frame}




\begin{frame}{Code p9}
    \begin{minipage}{0.95\textwidth}
        \centering
        \includegraphics[width=\textwidth]{code9}
    \end{minipage} 
    
\end{frame}

\begin{frame}{Code p10}
    \begin{minipage}{0.95\textwidth}
        \centering
        \includegraphics[width=\textwidth]{code10}
    \end{minipage} 
    
\end{frame}

\begin{frame}{Code p11}
    \begin{minipage}{0.95\textwidth}
        \centering
        \includegraphics[width=\textwidth]{code11}
    \end{minipage} 
    
\end{frame}

\begin{frame}{Code p12}
    \begin{minipage}{0.95\textwidth}
        \centering
        \includegraphics[width=\textwidth]{code12}
    \end{minipage} 
    
\end{frame}


\begin{frame}{Code p13}
    \begin{minipage}{0.95\textwidth}
        \centering
        \includegraphics[width=\textwidth]{code13}
    \end{minipage} 
    
\end{frame}

\begin{frame}{Code p14}
    \begin{minipage}{0.95\textwidth}
        \centering
        \includegraphics[width=\textwidth]{code14}
    \end{minipage} 
    
\end{frame}

\begin{frame}{Code p15}
    \begin{minipage}{0.95\textwidth}
        \centering
        \includegraphics[width=\textwidth]{code15}
    \end{minipage} 
    
\end{frame}

\begin{frame}{Code p16}
    \begin{minipage}{0.95\textwidth}
        \centering
        \includegraphics[width=\textwidth]{code16}
    \end{minipage} 
    
\end{frame}


\begin{frame}{Code p17}
    \begin{minipage}{0.95\textwidth}
        \centering
        \includegraphics[width=\textwidth]{code17}
    \end{minipage} 
    
\end{frame}


\begin{frame}{Code p18}
    \begin{minipage}{0.95\textwidth}
        \centering
        \includegraphics[width=\textwidth]{code18}
    \end{minipage} 
    
\end{frame}

\begin{frame}{Code p19}
    \begin{minipage}{0.95\textwidth}
        \centering
        \includegraphics[width=\textwidth]{code19}
    \end{minipage} 
    
\end{frame}


\begin{frame}{Code p20}
    \begin{minipage}{1.05\textwidth}
        \centering
        \includegraphics[width=\textwidth]{code20}
    \end{minipage} 
    
\end{frame}








\end{document}
